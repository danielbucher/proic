\section{Conclusão}
\label{sec:conclusao}

Ao longo deste trabalho, realizamos atividades que passam
pelas principais áreas de conhecimento da Engenharia de Software, como a
gerência de projetos, a gerência de configuração de software,
o levantamento e a elaboração de requisitos, a análise e o \textit{design},
a implementação, e aqui incluímos a criação de testes automatizados,
a manutenção e a implantação de uma plataforma real de software livre, dentre
outras.
%
Nos preocupamos em contribuir, na prática, com a escrita de código,
uma vez que, em nossa visão, esta é a principal missão de um Engenheiro de
Software.
%
Outra atividade realizada, e que julgamos ser de fundamental importância para a
continuidade deste trabalho, diz respeito ao repasse do conhecimento adquirido
para uma equipe de alunos do curso de Engenharia de Software, que trabalhou
inicialmente no Portal da UnB Gama.

Esta experiência nos proporcionou a possibilidade de participar de um processo
distribuído de desenvolvimento de software com outras equipes espalhadas pelo
país e de fazer parte de uma comunidade de software livre.
%
Ao final deste trabalho, julgamos que a rede Comunidade.UnB está bem próxima
da capacidade de ser lançada oficialmente para a universidade.
