\section{Fundamentação Teórica}
\label{sec:fundamentacao}

O princípio básico do ecossistema do software livre é promover a liberdade
do usuário, sem discriminar quem tem permissão para usar um software e seus
limites de uso, baseado na colaboração e num processo de desenvolvimento
aberto. Software livre é aquele que permite aos usuários usá-lo, estudá-lo,
modificá-lo e redistribui-lo, em geral, sem restrições para tal e prevenindo
que não sejam impostas restrições aos futuros usuários \cite{meirelles2013}.
Entendemos que este modelo de desenvolvimento de distribuição de software
seja o ideal em um estudo de caso envolvendo uma universidade, uma vez que
entendemos ser o papel da universidade não só promover a construção de
conhecimento, mas também a disseminação deste conhecimento para a sociedade,
e a utilização do software livre permite isso por causa dos direitos
que este fornece%
\footnote{Extraído de \url{http://www.gnu.org/} em Julho de 2014}.

Nesse contexto, o Noosfero é uma plataforma web livre para criação de redes sociais,com a proposta
de permitir aos usuários criarem sua própria
rede social personalizada, livre e autônoma.

O Noosfero foi desenvolvido na linguagem de programação Ruby
%
\footnote{\url{http://www.ruby-lang.org/en/}},
versão 1.8.7, e utiliza o \textit{framework} Model-View-Controller (MVC) para
aplicações web Ruby on Rails%
\footnote{\url{http://rubyonrails.org/}}, versão 2.3.5.
%
A alta capacidade produtiva que esse \textit{framework} possui por priorizar conceitos como
\textit{convention over configuration} (convenção antes de configuração)
e DRY\footnote{Uma forma de apologia ao reuso de código}
(\textit{Don't Repeat Yourself} - Não Repita a Si Mesmo), bem como, o alinhamento
entre a comunidade do Ruby on Rails com metodologias ágeis de
desenvolvimento de software, que são evidenciadas em uma série de ferramentas que
viabilizam o uso de práticas como TDD\footnote{Desenvolvimento orientado a testes}
e BDD\footnote{Design orientado a comportamento}, práticas adotadas no
desenvolvimento do Noosfero, e neste trabalho.

Além disso, a arquitetura do Noosfero foi pensada para permitir que este seja facilmente
expansível, de forma que funcionalidades que não sejam comuns ao conceito de
redes sociais sejam desenvolvidas como \textit{plugins}, assim diminuindo
o acoplamento e aumentando a coesão dos diversos módulos do sistema.

