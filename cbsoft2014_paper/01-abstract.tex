\begin{abstract}

Este trabalho de conclusão de curso apresenta os resultados de um estudo para viabilizar a implantação de uma rede de colaboração para a Universidade de  Brasília (UnB), que atue como um ambiente virtual para a criação e o  compartilhamento de conhecimento de forma colaborativa e horizontal. Para isso, escolhemos utilizar a plataforma  brasileira para redes sociais livres Noosfero, por entender que esta satisfaz as  necessidades imediatas do projeto, de acordo com estudos feitos pela  Universidade de São Paulo, quando adotou a mesma. Além da implantação em si na UnB, este estudo contemplou um levantamento de  requisitos e a implementação de um conjunto de funcionalidades e melhorias para a  plataforma em questão, de forma que atendesse as necessidades  básicas para podermos realizar estudos de caso com alunos da UnB Gama. Dessa forma, indicando como podemos oficializar a rede Comunidade.UnB, bem  como quais os próximos passos para que melhor atenda o público dessa universidade. Adicionalmente, os esforços e conhecimento adquiridos neste trabalho foram repassados  para uma equipe de desenvolvedores na UnB Gama, o que proporcionará a continuidade  e concretização da implantação desta rede na UnB, em 2014.

\end{abstract}
\vspace{1\baselineskip}

\begin{keywords}
Redes sociais, software livre, engenharia de software, Ruby on Rails.
\end{keywords}



