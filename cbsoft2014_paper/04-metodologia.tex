\section{Metodologia}
\label{sec:metodologia}

O processo de colaboração com o Noosfero inclui uma série de práticas apresentadas
pelas metodologias ágeis de desenvolvimento de software como o uso de testes
automatizados, a descrição das funcionalidades do projeto no formato de
histórias de usuário e a adoção de ferramentas para a utilização da metodologia
\textit{Behavior Driven Development} (BDD)%
\footnote{\url{http://en.wikipedia.org/wiki/Behavior-driven_development}},
uma evolução do \textit{Test Driven Development} (TDD)%
\footnote{\url{http://en.wikipedia.org/wiki/Test-driven_development}}
apresentada por Dan North \cite{north2006}.

Em suma, o processo de desenvolvimento de software utilizado durante este trabalho
também se baseia em metodologias ágeis, com a utilização de ciclos curtos de
forma a permitir que haja \textit{feedback} de forma mais rápida. A funcionalidades
desenvolvidas eram então submetidas \textit{upstream} e passam por uma revisão dos
\textit{commiters} oficiais do Noosfero antes de serem incorporadas, o que permite
que o desenvolvedor tenha \textit{feedback} sobre a qualidade de seu código e possa
melhorá-lo caso necessário.
