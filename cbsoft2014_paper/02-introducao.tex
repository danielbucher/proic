\section{Introdução}
\label{sec:intro}

Este artigo apresenta os resultados de um estudo de caso de implantação e
desenvolvimento de software livre na Universidade de Brasília (UnB). O
software livre em questão é o Noosfero\footnote{\url{http://noosfero.org/}},
uma plataforma para a criação de redes sociais livres criada pela
Cooperativa de Tecnologias Livres - Colivre%
\footnote{\url{http://colivre.coop.br/}}, uma empresa cooperativa que atua
exclusivamente com soluções livres. As contribuições para a comunidade do
Noosfero são feitas por alunos do Laboratório Avançado de Produção,
Pesquisa e Inovação em Software (LAPPIS)%
\footnote{\url{http://fga.unb.br/lappis/}} da UnB Gama desde junho de 2013
e produziram como resultado a implantação e a manutenção de dois ambiente
Noosfero na universidade: (\textit{i}) Comunidade.UnB\footnote{%
\url{http://comunidade.unb.br/}}, e o (\textit{ii}) Portal da UnB Gama%
\footnote{\url{http://fga.unb.br/}}. O primeiro é uma rede de colaboração
social para alunos, professores e funcionários técnico-administrativos da
UnB, ainda em estágio de homologação, e tem por objetivo fornecer um ambiente
virtual para a criação e o compartilhamento de conhecimento. O segundo é o
portal de informações e notícias da Faculdade do Gama (UnB/FGA).
%
Até a data da escrita deste artigo, os alunos do LAPPIS já contribuíram com
mais de vinte \textit{merge requests} incorporados no \textit{core} do
Noosfero e o Comunidade.Unb já possui 167 usuários e 24 comunidades.
