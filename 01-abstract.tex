\begin{abstract}

Neste texto apresentamos os resultados de um estudo de caso de implantação e colaboração de uma plataforma de software livre para a criação de redes sociais livres na Universidade de Brasília, de forma a disponibilizar um ambiente virtual que possibilite a criação e colaboração de conhecimento de forma horizontal e aberta. A colaboração com a plataforma se deu através da equipe de estagiários do Portal da UnB Gama, de Trabalhos de Conclusão de Curso e projetos de pesquisa na área de desenvolvimento colaborativo utilizando software livre.

\end{abstract}
\vspace{1\baselineskip}

\begin{keywords}
Redes sociais, software livre, engenharia de software, Ruby on Rails.
\end{keywords}



