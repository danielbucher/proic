\section{Resultados}
\label{sec:resultados}

A primeira funcionalidade desenvolvida surgiu da necessidade de se ter sub-%
comunidades para cada projeto da disciplina de Manutenção e Evolução de
Software da FGA, ministrada pelo Prof. Paulo Meirelles. O Noosfero já possuia
um \textit{plugin} que adicionava a possibilidade de criamos sub-comunidades
de determinada comunidade, no entanto este não oferecia uma forma clara do
usuário visualizar a relação entre estas. Como primeira colaboração deste
trabalho, desenvolvemos uma camada de visualização, dentro de uma comunidade,
na qual o usuário pudesse visualizar todas as comunidades ``filhas'' ou todas
as comunidades ``pais'' desta, uma vez que a relação entre comunidade e sub-%
comunidade é de N para N. Esta contribuição pode ser visualizada na comunidade
da disciplina mencionada%
\footnote{\url{http://comunidade.unb.br/profile/mes-fga/plugin/sub_organizations%
/children?display=full&type=community}}.

Outra contribuição interessante pro contexto da UnB, foi a melhoria no
\textit{plugin} de submissão de trabalhos, uma demanda que surgiu da
necessidade de existir um sistema web no qual os alunos pudessem enviar
seus trabalhos de conclusão de curso (TCC) e notificar seus orientadores e outros
interessados a respeito. Este \textit{plugin} permite não só que os alunos
façam upload de um arquivo, mas também que o mesmo seja versionado, criando
uma nova versão para cada novo upload realizado pelo mesmo aluno, além de
permitir a possibilidade de notificar pessoas interessadas via e-mail.
No primeiro semestre de 2014, foi realizado um teste%
\footnote{\url{http://fga.unb.br/tcc}} com os alunos do curso de engenharia de
software da FGA que estivessem fazendo TCC.

A \textit{wiki} do Participa.Br%
\footnote{\url{https://gitlab.com/participa/noosfero/wikis/CodeReview}},
um projeto de uma rede social livre para
possibilitar a participação social no meio virtual desenvolvido pela
presidência da república e que também contou com a contribuição dos alunos
do LAPPIS, possui uma série de \textit{patchs} (ou \textit{merge requests}
enviados no contexto deste trabalho, e de outros projetos do laboratório, e
já incorporado no núcleo do Noosfero.)
